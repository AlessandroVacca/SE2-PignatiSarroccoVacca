\section{Introduction}
eMall (e-Mobility for All) is an easy-to-use application which intent is to help the user to recharge their electric vehicle in order to reduce our carbon footprint.
Users need an application whose main intent is to plan the charging process of an electric vehicle, thus reducing the interference and constraints on our daily schedule.
Charging Point Operators need an application to manage their charging stations. 
\subsection{Purpose}

The aim of the product is to simplify the process of electric vehicle charging, improving the customers' experience. Moreover, the Charging Point Operators processes will be facilitated.
The experience will be enhanced because many aspects of the electric vehicle charging will be integrated, and they will be located within a single service.
\\\\
The application should provide two types of accesses for Customers (people who need to charge their vehicles) and Charging Point Operators.
\\\\
Customers will have the possibility to see which charging stations are available nearby and to also know the charging cost and if they have special offers.
They will also have the opportunity to book a charge at a station at a certain timeframe in advance and to start the charging process remotely.
They will be notified when the process has finished and the car is fully charged, and they will be able to pay from the application.
Moreover, the eMall service has a smart system that suggests to the customer the optimal charging solution, based on their schedule, the charging cost and the current state of charge.
On the other hand Charging Point Operators will be able to select energy options and set special offers.\\
The eMall platform will be composed of two modules, and thus the whole system will be developed modularly:
\begin{itemize}
        \item \textbf{Module 1: e-Mobility Service Provider subsystem:} 
        \\This part of the system will handle all the features regarding a customer, that will be interacting directly with this subsystem.
        \item \textbf{Module 2: Charging Point Management System subsystem:} 
        \\This part of the system will handle all the features regarding a single or multiple charging stations. e-Mobilty Service Providers and  check for socket availability and other various informations. Module 1 will interact with this module in order to check for socket availability and to actually start the charge. The Charging Point Operators will interact with the module to manually configure the stations options.
\end{itemize}
\subsubsection{Goals}
\begin{enumerate}[label=$\bullet$ \textbf{G\arabic*:}]
        \item \textbf{Allow customers to obtain information about nearby charging stations}
        \\
        The customer can view information about any nearby charging station,  
        such as charging cost and about special offers, availability of every type of charging socket, 
        and if a certain type of socket is occupied at the time,
         the estimated time for the first to be free.
        \item \textbf{Allow customers to book a charge for a certain timeframe}
        \\
        The customer can select any available charging station and book the charging for the next 15 minutes, if available. 
        \item \textbf{Allow customers to start the charge}
        \\
        The customer can remotely start the charge once the electric car has been connected to the station's socket, that must be manually unlocked from the service.
        \item \textbf{Allow customers to know when the charging has finished}
        \\
        The customer will be notified by the application when the charging of his vehicle has been completed.
        \item \textbf{Allow customers to pay for the charging service}
        \\
        The customer has the option to pay directly from the eMall application for the obtained service.
        \item \textbf{Allow customers to receive suggestions on where to charge}
        \\
        The customer can receive suggestions from eMall on the optimal station to charge, based on his schedule, his vehicle's state of charge and the stations' prices.
        %\item \textbf{Let the system to provide informations about charging stations}
        %\\
        
        \item \textbf{Allow Charging Point Operators to decide the energy acquisition options}
        \\
        The Charging Point Operator will have the possibility to choose whether to acquire energy automatically or to acquire it manually from a specific energy provider.
        Another option will be to store or use the energy from the batteries if they are present.
        \item \textbf{Allow Charging Point Operators to dynamically choose the charging cost and to set offers}
        \\
        The Charging Point Operator will have the possibility to select the charging cost price based on the energy providers prices. They will also be able to set special offers based on the energy providers prices.
\end{enumerate}

\subsection{Scope}
\subsubsection{Phenomena}
According to the paper "The World and the Machine" by M. Jackson and P. Zave, 
we can identify the application domains. 
The following table describes the world, shared and the machine phenomena, 
including the reference to which part controls the phenomena.
\begin{table}[!h]
        \centering
        \begin{tabular}{|c|c|c|}
        \hline
        \textbf{Phenomenon}                                        & \textbf{Controller} & \textbf{Shared} \\ \hline
        Customer charges his vehicle at a charging station  & W          & N      \\ \hline
        Customer books a charge at a charging station         & W          & Y      \\ \hline
        Customer connects the vehicle at the charging station              & W          & N      \\ \hline
        Customer receives QR code for his booking      & M          & Y      \\ \hline
        Customer authenticates with a QR code at the station at the selected timeslot      & M          & Y      \\ \hline
        Customer remotely starts the charge                   & W          & Y      \\ \hline       
        Customer registration                                 & M          & Y      \\ \hline
        Customer login                                        & W          & Y      \\ \hline
        Customer views a charging station's information       & W          & Y      \\ \hline
        Customer pays for the charge                          & W          & N      \\ \hline
        Customer receives advice on the optimal place to charge            & W          & Y      \\ \hline
        Customer is notified when the charge has been completed   & W          & Y      \\ \hline
        Charging Point Operator chooses from which energy provider to acquire energy   & W          & Y      \\ \hline
        Charging Point Operator decides to charge the station's battery    & W          & Y      \\ \hline
        Charging Point Operator select the cost for charging &W          & Y      \\ \hline
        Charging Point Operator login &W          & Y      \\ \hline
       \end{tabular}
\end{table}

\subsection{Definitions, Acronyms, Abbrevations}
\subsubsection{Definitions}
\begin{itemize}
        \item \textbf{Customers:} The people whom this service is directed. 
        They can belong to any age and gender. \label{Customer}
        Their goal is to efficiently charge their electric vehicle.        
        \item \textbf{Charging stations:} Places that offer the service of electric vehicle recharging.
        \item \textbf{e-Mobility Service Providers:} The companies that offer the service of charging at different stations.
        \item \textbf{Charging Point Operators:} The companies that manage charging stations (one or more).\label{CPO}
        \item \textbf{Distribution System Operators:} The charging station's energy providers.\label{DSO}
        \item \textbf{Charging Point Managment System:} The single CPO's system to manage all the IT infrastructure.
        \item \textbf{Notification:} It's an alert that a certain event occurred. 
        This alert can be a "Push Notification" on the smartphone, an SMS, an email and so on.
        \item \textbf{Push Notification:} It's an automated message sent by an application to a user when the application is not running.
\end{itemize}
\subsubsection{Acronyms}
\begin{itemize}
        \item \textbf{eMall:} e-Mobility for all.
        \item \textbf{eMSP:} e-Mobility Service Provider.
        \item \textbf{CPO:} Charging Point Operator.
        \item \textbf{DSO:} Distribution System Operator.
        \item \textbf{CPMS:} Charging Point Management System.
        \item \textbf{API:} Application Programming Interface.
        \item \textbf{UML:} Unified Modeling Language.
\end{itemize}
\subsubsection{Abbreviations}
\begin{itemize}
        \item \textbf{ID:} Identifier. It's a generally unique sequence of numbers or letters in order to unambiguously identify an entity.
        \item \textbf{Gn:} Goal number $n$
        \item \textbf{Dn:} Domain assumption number $n$
        \item \textbf{Rn:} Requirement number $n$
\end{itemize}
\subsection{Revision History}
\subsection{Reference Documents}
\begin{itemize}
        \item Specification document: "R\&DD Assignment A.Y. 2022-2023"
        \item Alloy official documentation: \underline{\url{https://alloytools.org/documentation.html}}
        \item Paper: "Jackson and Zave: the world and the machine"
        \item UML official specification \underline{\url{http://www.omg.org/spec/UML/}}
        \item BPMN official specification \underline{\url{http://www.omg.org/spec//BPMN/index.html}}
\end{itemize}
\subsection{Document Structure}
\begin{itemize}
        \item \textbf{Section 1: Introduction} \\This section offers a brief description of the problem and required functionalities.
        It also contains the list of definitions, acronyms and abbreviations that will be found in this document.
        Finally, there is the version history of the document, containing the revisions list and their content, and document structure, 
        which describes the main purposes of the sections of this document.
        \item \textbf{Section 2: Overall Description} \\This section shows the possible scenarios of interaction by the user with the system. It also offers a summarized description about the overall organization of the system, the hardware and software constraints and the interfaces needed to get it to work.
        It also contains a description of all the features offered by the application, and of the actors who use it.
        
        \item \textbf{Section 3: Specific Requirements} \\This section contains system requirements specification. It includes functional requirements described through some scenarios, use cases and sequence diagrams. 
        In this section non-functional requirements are specified too. Moreover, requirements are mapped to the goals of the system.

        \item \textbf{Section 4: Formal Analysis through Alloy} \\This section contains the description of the analysis' objective and the formal analysis with the use of Alloy.
        \item \textbf{Section 5: Effort Spent} \\This section presents the total effort spent by the project's members.
\end{itemize}

\section{Introduction}
eMall (e-Mobility for All) is an easy-to-use application which intent is to help the user to recharge their electric vehicle in order to reduce our carbon footprint.
Users need an application whose main intent is to plan the charging process of the electric vehicle, reducing the interference and constraints on our daily schedule.

\subsection{Purpose}

The aim of the product is to simplify the process of electric vehicle charging, improving the user experience of the customers.
The experience will be enhanced because many aspects of electric vehicle charging will be integrated and they will be located within a single service.
\\\\
Users will have the possibility to see which charging stations are available nearby and to also know the charging cost and if they have special offers.
They will also have the opportunity to book a charge at a station at a certain timeframe in advance and to start the charging process.
They will be notified when the process has finished and the car is fully charged, and they will be able to pay from the application.
Moreover, the eMall service has a smart system that suggests to the user the optimal charging solution, based on their schedule, the charging cost and the current state of charge.
\subsubsection{Goals}
\begin{enumerate}[label=$\bullet$ \textbf{G\arabic*:}]
        \item \textbf{Allow users to obtain information about nearby charging stations}
        \\
        The user can view information about any nearby charging station, 
        such as charging cost and about special offers, availability of every type of charging socket, 
        and if a certain type of socket is occupied at the time,
         the estimated time for the first to be free.
        \item \textbf{Allow users to book a charge for a certain timeframe}
        \\
        The user can select any available charging station and book the charging for a certain timeslot, if available. 
        \item \textbf{Allow users to start the charge}
        \\
        The user can remotely start the charge once the electric car has been connected to the station's socket.
        \item \textbf{Allow the users to know when the charging has finished}
        \\
        The user will be notified by the application when the charging of his vehicle has been completed.
        \item \textbf{Allow the users to pay for the charging service}
        \\
        The user has the option to pay directly from the eMall application for the obtained service.
        \item \textbf{Allow the users to receive suggestions on where to charge}
        \\
        The user can receive suggestions from eMall on the optimal station to charge, based on his schedule his vehicle's state of charge, the stations' prices and availability.
\end{enumerate}

\subsection{Scope}
\subsubsection{Phenomena}
According to the paper "The World and the Machine" by M.Jackson and P.Zave, 
we can identify the application domains. 
The following table describes the world, shared and the machine phenomena, 
including the reference to which part controls the phenomena.
\begin{table}[h]
        \centering
        \begin{tabular}{|c|c|c|}
        \hline
        \textbf{Phenomenon}                                        & \textbf{Controller} & \textbf{Shared} \\ \hline
        User charges his vehicle at charging stations  & W          & N      \\ \hline
        User books a charge at a charging station         & W          & Y      \\ \hline
        User connects the vehicle at the charging station              & W          & N      \\ \hline
        User receives QR code for his booking      & M          & Y      \\ \hline
        User authenticates with a QR code at the station at the selected timeslot      & M          & Y      \\ \hline
        User remotely starts the charge                   & W          & Y      \\ \hline       
        User registration                                 & M          & Y      \\ \hline
        User login                                        & W          & Y      \\ \hline
        User views a charging station's information       & W          & Y      \\ \hline
        User pays for the charge                          & W          & N      \\ \hline
        User receives advice on the optimal place to charge            & W          & Y      \\ \hline
        User is notified when the charge has been completed   & W          & Y      \\ \hline
       \end{tabular}
\end{table}

\subsection{Definitions, Acronyms, Abbrevations}
\subsubsection{Definitions}
\begin{itemize}
        \item \textbf{Users:} The people whom this service is directed. 
        They can belong to any age and gender. \label{Users}
        Their goal is to efficiently charge their electric vehicle.
        \item \textbf{Charging stations:} Places that offer the service of electric vehicle recharging.
        \item \textbf{e-Mobility Service Providers:} The companies that offer the service of charging at different stations.
        \item \textbf{Charging Point Operators:} The companies that manage charging stations (one or more).
        \item \textbf{Distribution System Operators:} The charging station's energy providers.
        \item \textbf{Charging Point Managment System:} The single CPO's system to manage all the IT infrastructure.
        \item \textbf{Notification:} It's an alert that a certain event occurred. 
        This alert can be a "Push Notification" on the smartphone, an SMS, an email and so on.
        \item \textbf{Push Notification:} It's an automated message sent by an application to a user when the application is not running.
\end{itemize}
\subsubsection{Acronyms}
\begin{itemize}
        \item \textbf{eMall:} e-Mobility for all.
        \item \textbf{eMSP:} e-Mobility Service Provider.
        \item \textbf{CPO:} Charging Point Operator.
        \item \textbf{DSO:} Distribution System Operator.
        \item \textbf{CPMS:} Charging Point Managment System.
        \item \textbf{API:} Application Programming Interface.
        \item \textbf{UML:} Unified Modeling Language.
\end{itemize}
\subsubsection{Abbreviations}
\begin{itemize}
        \item \textbf{ID:} Identifier. It's a generally unique sequence of numbers or letters in order to unambiguously identify an entity.
        \item \textbf{Gn:} Goal number $n$
        \item \textbf{Dn:} Domain assumption number $n$
        \item \textbf{Rn:} Requirement number $n$
\end{itemize}
\subsection{Revision History}
\subsection{Reference Documents}
\begin{itemize}
        \item Specification document: "R\&DD Assignment A.Y. 2022-2023"
        \item Alloy official documentation: \underline{\url{https://alloytools.org/documentation.html}}
        \item Paper: "Jackson and Zave: the world and the machine"
        \item UML official specification \underline{\url{http://www.omg.org/spec/UML/}}
        \item BPMN official specification \underline{\url{http://www.omg.org/spec//BPMN/index.html}}
\end{itemize}
\subsection{Document Structure}
\begin{itemize}
        \item \textbf{Section 1: Introduction} \\This section offers a brief description of the problem and required functionalities.
        It also contains the list of definitions, acronyms and abbreviations that will be found in this document.
        Finally, there is the version history of the document, containing the revisions list and their content, and document structure, 
        which describes the main purposes of the sections of this document.
        \item \textbf{Section 2: Overall Description} \\This section shows the possible scenarios of interaction by the user with the system. It also offers a summarized description about the overall organization of the system, the hardware and software constraints and the interfaces needed to get it to work.
        It also contains a description of all the features offered by the application, and of the actors who use it.
        
        \item \textbf{Section 3: Specific Requirements} \\This section contains several visual mockups in order to explain the interfaces listed in Section 2. 
        It also contains a description of functional requirements through use cases and diagrams.

        \item \textbf{Section 4: Formal Analysis through Alloy} \\This section contains the description of the analysis' objective and the formal analysis with the use of Alloy.
        \item \textbf{Section 5: Effort Spent} \\This section presents the total effort spent by the project's members.
\end{itemize}

\section{Introduction}
eMall (e-Mobility for All) is an easy-to-use application which intent is to help the user to recharge their electric vehicle in order to reduce our carbon footprint. Users need an application whose main intent is to plan the charging process of an electric vehicle, thus reducing the interference and constraints on our daily schedule. Charging Point Operators need an application to manage their charging stations.
\subsection{Purpose}
The aim of the product is to simplify the process of electric vehicle charging, improving the customers’ experience. Moreover, the Charging Point Operators processes will be facilitated. The experience will be enhanced because many aspects of the electric vehicle charging will be integrated, and they will be located within a single service.\newline
The purpose of this document is to provide an exhausting explanation about eMall, focusing in particular on the architecture that will be adopted, the modules of the system and their interfaces.
Furthermore, a runtime view of the core functionalities of the product are provided, accompanied by some detailed interactions diagrams that show the message exchanging between the components.
Finally, there are mentions about the implementation, testing and integration processes.
\subsection{Scope}
The application provides different actors that will use the application, which are customers and CPOs.

\textbf{Customers} can access the mobile application in order to see the nearby charging stations and to book a charge, selecting the type of socket. Moreover, they can see all the booking's details and unlock the correspondent socket and start the charging.  Finally, they can pay directly on the application, if they want.
They can also change the vehicle with a new one.

\textbf{CPOs} can access the web application in order to see the managed charging stations and they can set special offers and change the costs of the different sockets. They can manage the charging batteries and set their policies. Finally they see the different DSOs for every station and they can choose if they want to acquire energy from them.


\subsection{Definitions, Acronyms, Abbreviations}
\subsubsection{Definitions}
\begin{itemize}
    \item \textbf{Client-side scripting:} it is performed to generate a code that can run on the client end (browser) without needing the server side processing.
    \item \textbf{Code On Demand:} in distributed computing, it is any technology that sends executable software code from a server computer to a client computer upon request from the client's software.
    \item \textbf{Middleware:} in distributed applications, it represents the software that enables communication and management of data.
    \item \textbf{RESTful:} it's a software architectural style that defines a set of constraints to be used for creating Web services.
    \item \textbf{e-Mobility Service Providers:} The companies that offer the service of charging at different stations.
    \item \textbf{Charging Point Operators:} The companies that manage charging stations (one or more).\label{CPO}
    \item \textbf{Distribution System Operators:} The charging station's energy providers.\label{DSO}
    \item \textbf{Charging Point Management System:} The single CPO's system to manage all the IT infrastructure.
    \item \textbf{Tier:} it is a row or layer in a series of similarly arranged objects. In computer programming, the parts of a program can be distributed among several tiers, each located in a different computer in a network.
    \item \textbf{Module 1:} this represents the eMSP subsystem explained in the RASD document.
    \item \textbf{Module 2:} this represents the CPMS subsystem explained in the RASD document.
\end{itemize}

\subsubsection{Acronyms}
\begin{itemize}
    \item \textbf{eMall:} e-Mobility for all
    \item \textbf{eMSP:} e-Mobility Service Provider
    \item \textbf{CPO:} Charging Point Operator
    \item \textbf{DSO:} Distribution System Operator
    \item \textbf{CPMS:} Charging Point Management System
    \item \textbf{API:} Application Programming Interface, it indicates on demand procedure which supply a specific task.
    \item \textbf{DBMS:} Database Management System.
    \item \textbf{DD:} Design Document    
    \item \textbf{HTTPS:} Hypertext Transfer Protocol Secure (HTTPS) is an extension of the Hypertext Transfer Protocol (HTTP). It is used for secure communication over a computer network, and is widely adopted on the Internet.
    \item \textbf{MVP:} Minimum Viable Product, it is a version of a product with just enough features to be usable by early customers who can then provide feedback for future product development.
    \item \textbf{RASD:} Requirements Analysis and Specification Document
    \item \textbf{S2B:} Software to Be, it is the one designed in this document and not yet implemented.
    \item \textbf{TLS:} Transport Layer Security, it is a protocol which aims primarily to provide privacy and data integrity between two or more communicating computer applications
\end{itemize}
\subsubsection{Abbreviations}
\begin{itemize}
    \item \textbf{Gn:} Goal number $n$
    \item \textbf{Dn:} Domain assumption number $n$
    \item \textbf{Rn:} Requirement number $n$
\end{itemize}
\subsection{Revision History}
\begin{itemize}
    \item January 8, 2022: version 1.0, initial release
    \item January 8, 2022: version 1.1, updated runtime views (latest release)
\end{itemize}
\subsection{Reference Documents}
\begin{itemize}
    \item Requirements Analysis Specification Document (RASD)
    \item UML official specification: \underline{\url{https://www.omg.org/spec/UML/}}
\end{itemize}
\subsection{Document Structure}
\begin{itemize}
    \item \textbf{Section 1: Introduction}\\This section offers a brief description of the document that will be presented, with all the definitions, acronyms and abbreviations that will be found reading it.
    \item \textbf{Section 2: Architectural Design}\\This section is addressed to the developer team and offers a more detailed description of the architecture of the system. The first part describes the chosen paradigm and the overall split of the system into several layers. Furthermore, an high-level description of the system is provided, together with a presentation of the modules composing its nodes. Finally, there is a concrete description of the tiers forming the S2B.
    \item \textbf{Section 3: User Interface Design}\\This section is useful for graphical designers of the S2B and contains several mockups of the application, together with some charts useful to understand the correct flow of execution of it. The presented mockups refers to the client-side experience.
    \item \textbf{Section 4: Requirements Traceability}\\This section acts as a bridge between the RASD and DD document, providing a complete mapping of the requirements and goals described in the RASD to the logical modules presented in this document.
    \item \textbf{Section 5: Implementation, Integration and Test Plan}\\The last section is again addressed to the developer team and describes the procedures followed for implementing, testing and integrating the components of our S2B. There will be a detailed description of the core functionalities of it, together with a complete report about how to implement and test them.
\end{itemize}

\section{Introduction}
eMall (e-Mobility for All) is an easy-to-use application which intent is to help the user to recharge their electric vehicle in order to reduce our carbon footprint. Users need an application whose main intent is to plan the charging process of an electric vehicle, thus reducing the interference and constraints on our daily schedule. Charging Point Operators need an application to manage their charging stations.
\subsection{Purpose}
The aim of the product is to simplify the process of electric vehicle charging, improving the customers’ experience. Moreover, the Charging Point Operators processes will be facilitated. The experience will be enhanced because many aspects of the electric vehicle charging will be integrated, and they will be located within a single service.\newline
The purpose of this document is to provide an exhausting explanation about eMall, focusing in particular on the architecture that will be adopted, the modules of the system and their interfaces.
Furthermore, a runtime view of the core functionalities of the product are provided, accompanied by some detailed interactions diagrams that show the message exchanging between the components.
Finally, there are mentions about the implementation, testing and integration processes.
\subsection{Scope}
\subsection{Definitions, Acronyms, Abbreviations}
\subsubsection{Definitions}
\subsubsection{Acronyms}
\subsubsection{Abbreviations}
\begin{itemize}
    \item \textbf{Gn:} Goal number $n$
    \item \textbf{Dn:} Domain assumption number $n$
    \item \textbf{Rn:} Requirement number $n$
\end{itemize}
\subsection{Revision History}
\subsection{Reference Documents}
\begin{itemize}
    \item Requirements Analysis Specification Document (RASD)
    \item UML official specification: \underline{\url{https://www.omg.org/spec/UML/}}
\end{itemize}
\subsection{Document Structure}
\begin{itemize}
    \item \textbf{Section 1: Introduction}\\This section offers a brief description of the document that will be presented, with all the definitions, acronyms and abbreviations that will be found reading it.
    \item \textbf{Section 2: Architectural Design}\\This section is addressed to the developer team and offers a more detailed description of the architecture of the system. The first part describes the chosen paradigm and the overall split of the system into several layers. Furthermore, an high-level description of the system is provided, together with a presentation of the modules composing its nodes. Finally, there is a concrete description of the tiers forming the S2B.
    \item \textbf{Section 3: User Interface Design}\\This section is useful for graphical designers of the S2B and contains several mockups of the application, together with some charts useful to understand the correct flow of execution of it. The presented mockups refers to the client-side experience.
    \item \textbf{Section 4: Requirements Traceability}\\This section acts as a bridge between the RASD and DD document, providing a complete mapping of the requirements and goals described in the RASD to the logical modules presented in this document.
    \item \textbf{Section 5: Implementation, Integration and Test Plan}\\The last section is again addressed to the developer team and describes the procedures followed for implementing, testing and integrating the components of our S2B. There will be a detailed description of the core functionalities of it, together with a complete report about how to implement and test them.
\end{itemize}
